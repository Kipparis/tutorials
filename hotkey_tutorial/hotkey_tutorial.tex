\documentclass{article}
\usepackage{base}

\begin{document}
\tableofcontents
\newpage
\section{Общее}
\begin{itemize}
    \item \verb|mod+shft+;| - открыть мануал
\end{itemize}

\section{Запись}
\subsection{Скришноты}
\begin{itemize}
    \item \verb|prnt_scr| - создает окошко с выделенной областью
    \item \verb|mod+ctrl+prnt_scr| - копирует выбранную область в буффер
        обмена
    \item \verb|mod+shft+prnt_scr| - сохраняет выбранную область в папку
        для картинок по умолчанию
\end{itemize}

\subsection{Аудио, видео}
Запускаешь скрипт с помощью \verb|mod+ctrl+j|, туда вбиваешь любую
комбинацию букв, каждая из которых обозначает:
\begin{itemize}
    \item \verb|v| - запись видео с экрана (открывает \verb|obs|)
    \item \verb|p| - запись аудио (playback)
    \item \verb|m| - запись с микрофона (на каждом компьютере надо
        тестировать)
\end{itemize}
\verb|mod+ctrl+k| - остановить запись

\section{Фокусировка}
\subsection{Таймер}
\begin{itemize}
    \item \verb|mod+t| - создать таймер. Параметры:
        \subitem *время (в минутах)
        \subitem *кол-во звоночков
        \subitem текст, который будет выводиться
        \subitem файл со звоночком
    \item \verb|mod+shft+t| - остановить таймер
\end{itemize}


\section{Рабочее пространство}
\subsection{Управление границами}
\begin{itemize}
    \item \verb|mod+shft+g| - расстояние между окнами
    \item \verb|mod+b| - границы окон
\end{itemize}


\subsection{Создание новых программ}
С помощью горячих клавиш:
\begin{itemize}
    \item \verb|mod+ret| - открытие терминала \verb|st|
    \item \verb|mod+ctrl+ret| - запускает терминал \verb|urxvt| с
        дефолтной оболочкой (для случая когда намудрил с конфигами)
\end{itemize}

\par Горячие клавиши + выделение области:
\begin{itemize}
    \item \verb|mod+shft+s| - создается терминал \verb|st| размером с
        выбранную область
    \item \verb|mod+shft+w| - создается терминал \verb|st| с открытым
        менеджером \verb|ranger| размером с выбранную область
    \item \verb|mod+shft+a| - открывает браузер \verb|surf|
\end{itemize}

\section{Внешние устройства}
\begin{itemize}
    \item \verb|mod+m| - смонтировать устройство из списка доступных
    \item \verb|mod+u| - отмонтировать устройство из списка доступных
\end{itemize}

\section{Управление}
\subsection{Система}
\begin{itemize}
    \item \verb|mod+ctrl+b| - открывает \verb|bmenu| (CLI configuration tool)
много информации о системе.  Если её недостаточно, то можно искать
в конфиге \verb|zsh|.
    \item \verb|mod+shft+x| - выходит в режим логина (надо ввести пароль
        чтобы выйти)
\end{itemize}

\subsection{Аудио}
\verb|mod+ctrl+m| - открывает \verb|pavucontrol| - микшер

\section{Приложения}
\begin{itemize}
    \item \verb|dunst|, \verb|notify-send| - нотификации
    \item \verb|st| - терминал
    \item \verb|dmenu| - интеракция с пользователем
    \item \verb|wal| - (pywal) управление цветовой схемой
        \subitem когда ставишь новую схему, придется много повозиться
        чтоб остальные приложения подходили
\end{itemize}


\end{document}

