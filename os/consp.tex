% \documentclass{article}[20pt]
\documentclass[12pt]{extarticle}



%Russian-specific packages
%--------------------------------------
\usepackage[T2A]{fontenc}
\usepackage[utf8]{inputenc}
\usepackage[russian]{babel}
%--------------------------------------

%Hyphenation rules
%--------------------------------------
\usepackage{hyphenat}
\hyphenation{ма-те-ма-ти-ка вос-ста-нав-ли-вать}
%--------------------------------------

\setlength{\parindent}{0em} % space before par
\setlength{\parskip}{1em}   % spacing between pars

\usepackage{enumitem} % configure lists
% \setlist{noitemsep,topsep=0pt} % separation
\setlist{nosep} % separation
\setlist[itemize,1]{label=$\triangleleft$} % triangles looking to the left

% margin from paper borders
\usepackage[margin={0.8in, 0.3in}]{geometry}

\usepackage{amsmath}
\usepackage{amssymb}
\usepackage{amsfonts}
% 1 - Name
% 2 - Goes to the index
% 3 - Definition
% internal - some text about teorem
% \newenvironment{theorem}[3]{
% beginning
%     \vspace{0.5\parindent}
%     \par\index{#2}\emph{\underline{Th.} \textbf{#1} #3.}  % teorem's name
%     \par
% }{
% ending
    % \vspace{0.2\parindent}
% }

% 1 - Name
% 2 - Goes to index
% internal - Name's definition
\newenvironment{mydef}[2]{
% beginning
    \vspace{0.5\parindent}
    \par \index{#2} \emph{\underline{Def.} \textbf{#1}} -
}{
% ending
    \vspace{0.2\parindent}
}

% TODO
% \newcommand{hidefinition}

% 1 - Name
% 2 - Given
% 3 - Task
% internal - solution
\newenvironment{example}[3]{
    \vspace{0.5\parindent}
    \par\emph{\underline{Ex.}} #1
    \par \textbf{Given:} #2
    \par \textbf{Find:} #3
    \par \textbf{Solution:}
}{
    \vspace{0.5\parindent}
}

% displaystyle в каждом мас моде
\everymath{\displaystyle}

% TODO: command for drawing box around text (so I can see why some artifact are

% TODO:
% \newcommand{grad}

% indent first line after section
\usepackage{indentfirst}

\usepackage{makeidx}
\makeindex

% keep this in the very bottom
% \usepackage[pdftex]{hyperref}
% remove red boxes around links
\usepackage[pdftex,pdfborderstyle={/S/U/W 0}]{hyperref} % this disables the boxes around links


\begin{document}

\tableofcontents
\printindex

\section{Процессы и потоки}
\paragraph{Планировщик процессов}

\section{Управление памятью}
\paragraph{Планировщик памяти}

\section{Файловые системы}
\paragraph{Журналируемая файловая система}
\index{Журналируемая файловая система}
Основной принцип заключается в журналировании всех намерений файловой
системы перед их осуществлением.  В случае сбоя системы, система может
посмотреть в журнал, определить что она собиралась сделать на момент
аварии, и завершить свою работу.
Позволяет сохранить целостность файловой системы при сбоях.
\par В линуксе:
\begin{itemize}
    \item \verb|NTFS|
    \item \verb|ext4|
    \item \verb|btrfs| - позволяет создавать RAID-массивы, подтома,
        снимки, сжимать данные на лету без использования сторонних
        инструментов. Считается стабильной с версии ядра Linux 4.3.1
\end{itemize}
\par При журналировании все операции должны быть
\textbf{идемпонентными}\index{Идемпонентная операция} -
возможность их повторения необходимое число раз без нанесения
какого-либо вреда. Для дополнительно надежности может быть реализована
концепция \textbf{атомарной транзакции}\index{Атомарная транзакция}
(либо выполняются все операции, либо не одна из них)

% TODO: плюсы и минусы файловых систем

\paragraph{Виртуальные файловые системы} или VFS.
Выделяем какую-то часть файловой системы, которая является общая
для всех, помещаем ее на отдельные уровень, из которого вызываются
расположенные ниже уже конкретные файловые системы (делаем так, чтобы
файловые системы предоставляли нам определенный интерфейс).


\begin{center}
    \bf Хранение свободных блоков
\end{center}

В виде битовый массив: плюсы, минусы
\par В виде связанного списка: плюсы, минусы

\par Плюсы и минусы размеров блоков памяти

\section{Диски}
Дефрагментацию для hdd делать полезно, т.к. при прокрутке диска мы будет
последовательно считывать информацию. А вот дефрагментацию ssd делать
вредно, т.к. ssd основан на flash памяти, которая после какого-либо
кол-ва применений портится.

\end{document}
